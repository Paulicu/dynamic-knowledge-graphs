\section{Introduction}

Knowledge Graphs (KGs) \cite{hogan2021} serve as a powerful tool to organize and represent structured knowledge across various domains, including the semantic web and artificial intelligence. By encoding knowledge as triples, KGs effectively capture the relationships between entities and enable reasoning over interconnected data. Despite their widespread utility, traditional static KGs face significant limitations. Their inability to adapt to the dynamic nature of real-world information, where facts and relationships are in constant flux, restricts their capacity to fully represent evolving data. 

This limitation of static KGs has motivated the development of Dynamic Knowledge Graphs (DKGs), which incorporate a temporal dimension to account for the evolving nature of data. DKGs enable the representation of changes in entities, relationships, and facts over time.

However, the integration of temporal information into KGs introduces several complexities. Representing time-sensitive data and modeling transitions between different states are non-trivial challenges that demand sophisticated solutions. Additionally, tasks like Knowledge Graph Completion (KGC) and Entity Alignment (EA), which are already challenging in static settings, become significantly more complex in dynamic environments. Ensuring consistency and accuracy in these evolving contexts requires innovative approaches that can effectively handle the temporal dimension.

To address these challenges, neurosymbolic methods have emerged as 
a promising avenue. These methods combine the strengths of symbolic reasoning, which provides a structured understanding of knowledge, with the adaptability and scalability 
of neural networks. Neurosymbolic approaches enable more effective reasoning over dynamic and temporal KGs, with tasks such as temporal KGC and EA by leveraging both logical structure and data-driven learning.

The research article \cite{alam2024} that I have chosen to analyze introduces a comprehensive exploration of DKGs and highlights the challenges of incorporating temporal information. It reviews existing methods for KGC and EA in dynamic settings and proposes neurosymbolic approaches to improve the handling of temporal data. I aim to extend this topic using the article as starting point.

I decided to organize the remainder of this paper as follows. Section 2 provides a summary of the chosen research article. Section 3 presents an extended literature survey, delving into additional works and perspectives in the field. Section 4 outlines my proposal, building on the foundations set by the research article and proposing new directions for advancing this domain.