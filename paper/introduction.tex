\section{Introduction}

Knowledge Graphs (KGs) - Intro Paragraph

Assigned Paper \cite{alam2024} - Intro Paragraph

The assigned paper \cite{alam2024} provides a foundational exploration of DKGs, introducing their structure, challenges, and potential applications. It begins with a detailed overview of preliminaries, defining various types of KGs, including static, temporal, and dynamic KGs, with illustrative examples. By distinguishing these types, the paper establishes a clear framework for understanding how temporal dynamics affect the design and use of knowledge graphs.

A key focus of the paper is on techniques for representing temporal information within both temporal and dynamic KGs. These techniques range from basic uses of XML Schema Definition (XSD) types to more advanced methods such as temporal properties, reification, time ontology in OWL, named graphs and quadruples, RDF-star, and versioning. Each of these approaches is discussed in detail, accompanied by a comparative analysis of their strengths, weaknesses, and suitability for different scenarios.

The paper also highlights prominent knowledge graphs, such as DBpedia, YAGO, Wikidata, and EventGraph, examining how these widely-used systems handle temporal information. This provides valuable insights into practical implementations and the trade-offs made by developers when designing temporal and dynamic KGs.

In addition, the paper explores methodologies for Dynamic KG Completion (KGC) and Entity Alignment (EA), two critical tasks in the field. For dynamic KGC, the methods are categorized into temporal KG completion techniques and non-temporal dynamic KG completion techniques. For EA, the paper delves into both temporal EA methods and those that address evolving EA scenarios, shedding light on the challenges posed by dynamic and temporal contexts.

Finally, the paper concludes with an analysis of existing research gaps and outlines potential directions for future work. These include enhancing the scalability of DKG algorithms to accommodate the vast and ever-growing size of modern knowledge graphs, integrating more diverse types of information, and exploring the role of Large Language Models (LLMs) in DKG tasks.

Purpose of my Paper - Paragraph

(Maybe) Paper Structure - Paragraph
