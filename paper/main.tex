\documentclass[runningheads]{llncs}

\usepackage[T1]{fontenc}

\begin{document}
    \title{Dynamic Knowledge Graphs: A Review of Current Literature and Research Proposals}
    \titlerunning{Dynamic KGs: Literature Review and Research Proposals}

    \author{Paul-Adrian Uif\u{a}lean}
    \authorrunning{Paul Uif\u{a}lean}

    \institute {Babe\c{s}-Bolyai University}
    
    \maketitle 
    
    \begin{abstract}
        This paper explores the complexities of Dynamic Knowledge Graphs (DKGs) in the context Knowledge Graph Completion (KGC), and Entity Alignment (EA). 
        It begins with a summary of a recent research article that establishes a foundation for understanding DKGs and their applications. 
        The paper then examines the current literature on DKG techniques, addressing key limitations such as the overreliance on graph-based representations and the insufficient handling of 
        temporal and multimodal data.
        In the proposal section, the paper suggests several directions for future research, such as incorporating ontological models, improving the integration of multimodal literals within 
        dynamic contexts, and utilizing Large Language Models (LLMs) for enhanced temporal reasoning. These approaches aim to improve the adaptability, accuracy, and scalability of DKG 
        algorithms, leading to more effective solutions for dynamic data processing and knowledge representation in real-world applications.
    \end{abstract}
    
    \keywords{Dynamic Knowledge Graphs, Entity Alignment, Knowledge Graph Completion,  Temporal Reasoning, Large Language Models, Multimodal Data Integration}

    \section{Introduction}

Knowledge Graphs (KGs) \cite{hogan2021} serve as a powerful tool to organize and represent structured knowledge across various domains, including the semantic web and artificial 
intelligence. By encoding knowledge as triples, KGs effectively capture the relationships between entities and enable reasoning over interconnected data. Despite their widespread 
utility, traditional static KGs face significant limitations. Their inability to adapt to the dynamic nature of real-world information, where facts and relationships are in constant 
flux, restricts their capacity to fully represent evolving data. 

This limitation of static KGs has motivated the development of Dynamic Knowledge Graphs (DKGs), which incorporate a temporal dimension to account for the evolving nature of data. 
DKGs enable the representation of changes in entities, relationships, and facts over time.

However, the integration of temporal information into KGs introduces several complexities. Representing time-sensitive data and modeling transitions between different states are 
non-trivial challenges that demand sophisticated solutions. Additionally, tasks like Knowledge Graph Completion (KGC) and Entity Alignment (EA), which are already challenging in 
static settings, become significantly more complex in dynamic environments. Ensuring consistency and accuracy in these evolving contexts requires innovative approaches that can 
effectively handle the temporal dimension.

To address these challenges, neurosymbolic methods have emerged as a promising avenue. These methods combine the strengths of symbolic reasoning, which provides a structured 
understanding of knowledge, with the adaptability and scalability of neural networks. Neurosymbolic approaches enable more effective reasoning over dynamic and temporal KGs, with 
tasks such as temporal KGC and EA by leveraging both logical structure and data-driven learning.

The research article \cite{alam2024} that I have chosen to analyze introduces a comprehensive exploration of DKGs and highlights the challenges of incorporating temporal information. 
It reviews existing methods for KGC and EA in dynamic settings and proposes neurosymbolic approaches to improve the handling of temporal data. 
I aim to extend this topic using the article as starting point.

I decided to organize the remainder of this paper as follows. Section 2 provides a summary of the chosen research article. Section 3 presents an extended literature survey, 
delving into additional works and perspectives in the field. Section 4 outlines my proposal, building on the foundations set by the research article and proposing new directions for 
advancing this domain.
    \section{Research Summary}

The research article under review offers an extensive overview of various methods for dynamic Knowledge Graph Completion (KGC) and Entity Alignment (EA), covering both temporal and non-temporal approaches. Additionally, the paper delves into techniques for the representation of temporal information in Knowledge Graphs (KGs) and discusses the complexities associated with the evolution of dynamic data. This section summarizes the core concepts and findings presented in the paper.

\subsection{Types of Knowledge Graphs}

The paper begins by establishing a clear distinction between the three main types of Knowledge Graphs, which serve as foundational concepts in understanding dynamic data 
representation.

\textbf{Static Knowledge Graphs} represent a fixed, unchanging set of facts that do not accommodate evolving or time-sensitive data. Static KGs are well-suited for situations where 
the facts are constant and do not change over time. An example of a static KG would be one stating that "Barack Obama is the president of the USA". Here, the knowledge remains fixed 
without the need to incorporate time-based updates.

\textbf{Temporal Knowledge Graphs (TKGs)} extend static KGs by incorporating time stamps to represent the validity of facts over specified periods. This enables the inclusion of 
time-based queries and allows for the modeling of changing facts over time. For instance, a TKG may indicate that "Barack Obama was president from 2009 to 2017", providing temporal 
context for the fact.

\textbf{Dynamic Knowledge Graphs (DKGs)} are the most complex form of KGs, representing a series of evolving knowledge states indexed by time steps. Each snapshot of the graph can 
either be static or temporal, and transitions between these snapshots capture the evolution of entities, relationships, and facts over time. For example, a DKG could represent the 
sequence of presidents of the USA over time, with each snapshot capturing a different president and the associated time period in office.

\subsection{Techniques for Representing Temporal Information}

The paper emphasizes the challenges present in modeling dynamic data and introduces various techniques for incorporating temporal information into Knowledge Graphs. 
One of the core issues is that dynamic data requires methods for tracking changes over time while maintaining the consistency of the graph's structure.

One common approach is the use of \textbf{Temporal Properties}, where XML Schema Definition (XSD) data types such as \texttt{xsd:dateTime}, \texttt{xsd:duration}, and \texttt{xsd:date} 
are utilized to explicitly encode temporal aspects within KGs. By representing facts and relationships with these time-related data types, KGs can support time-based queries and 
reasoning, allowing the exploration of time-dependent information.

\textbf{Reification} is another technique that extends the triple structure of KGs by treating a triple (subject, predicate, object) as the subject of a new triple. This enables the 
attachment of additional metadata, including temporal information, to the original triple. As a result, reification allows for the modeling of evolving facts while maintaining the 
flexibility to include context, such as when the fact holds true.

The \textbf{Time Ontology in OWL} defines temporal concepts, such as instants, intervals, and durations, through a set of classes and properties. This ontology offers a structured 
approach for representing time within an OWL-compatible framework, facilitating reasoning over temporal relationships and providing a formal mechanism for incorporating time-based 
knowledge.

Another method, \textbf{Named Graphs and Quadruples}, organizes triples into named graphs, which are essentially groups of triples. To incorporate temporal information, a fourth 
element representing the temporal context is added to each graph, providing a flexible way to manage time-sensitive data within a KG. This technique allows for the association of 
temporal data with specific snapshots of the graph.

\textbf{RDF-star} is an extension of the Resource Description Framework (RDF) format that enhances the expressiveness of triples by enabling the inclusion of metadata, such as 
temporal information, in a more compact and efficient manner. This compactness allows RDF-star to represent temporal relationships between entities and facts without sacrificing 
clarity or detail, improving performance in handling complex temporal queries.

Finally, \textbf{Versioning} involves tracking changes to facts or entities by explicitly storing their different versions over time. This technique is particularly valuable for 
creating historical records of knowledge, which can be queried for insights like trend analysis or anomaly detection, thus adding an important dimension to the representation of 
evolving knowledge.

\subsection{Comparison of Temporal and Dynamic Capabilities Across RDF Techniques}

The paper compares different RDF techniques in terms of their temporal and dynamic capabilities. A key contribution of the paper is the comparison table that evaluates various 
methods based on their ability to handle temporal and dynamic data. The following table summarizes this comparison.

\begin{table}[h]
    \centering
    \caption{Comparison of Temporal and Dynamic Capabilities Across Various RDF Techniques}
    \begin{tabular}{|l|c|c|}
        \hline
        \textbf{RDF Technique}           & \textbf{Temporal} & \textbf{Dynamic}  \\ \hline
        Temporal Properties              & X                 &                   \\ \hline
        Reification                      & X                 & X                 \\ \hline
        Time Ontology in OWL             & X                 &                   \\ \hline
        Named Graphs and Quadruples      & X*                & X*                \\ \hline
        RDF-star                         & X                 & X                 \\ \hline
        Versioning                       &                   & X                 \\ \hline
    \end{tabular}
\end{table}

Upon analyzing each technique, I find that RDF-star and versioning stand out as most versatile, as they cater to both temporal and dynamic data, making them suitable for a wide range
of applications. These methods seem to strike a balance between functionality and simplicity, allowing for more flexible representations of evolving data without excessive complexity.

However, techniques like Temporal Properties and Time Ontology in OWL are more specialized, focusing solely on the temporal aspect. While these methods are essential for tasks 
requiring detailed temporal reasoning, they may not be as flexible when dealing with purely dynamic data that evolves without a clear temporal component.

The asterisk under Named Graphs and Quadruples is a useful note, because it indicates that a choice must be made between temporal and dynamic, depending on the semantics intended 
by the graph owner. This adds an important nuance, as it emphasizes the trade-offs involved when designing KGs for specific applications.

\subsection{Prominent Knowledge Graphs and Their Temporal Representations}

The paper also discusses various well-known Knowledge Graphs and their strategies for handling temporal data. These include large-scale KGs such as Wikidata and YAGO, which have 
adopted different methods for integrating time information into their structures. These case studies illustrate how temporal information is represented and managed within practical 
systems, providing insight into the strengths and weaknesses of various approaches.

The authors also highlight the challenges faced by these KGs in managing the evolution of data over time, particularly in relation to entity alignment and knowledge graph completion. 
By examining these case studies, the paper offers valuable lessons in how real-world KGs deal with the complexities of dynamic and temporal data.

\subsection{Methods for Dynamic Knowledge Graph Completion}

In the dynamic context, the paper identifies two primary categories of methods for Knowledge Graph Completion (KGC), which aim to predict missing facts within dynamic KGs.

\textbf{Temporal KG Completion} focuses on predicting missing facts in KGs with temporal data. This includes tasks such as temporal link prediction, where the goal is to predict the 
presence of missing temporal facts, and event forecasting, which involves predicting the occurrence of future events based on historical data.

\textbf{Non-Temporal Dynamic KG Completion} deals with predicting missing facts in KGs without explicit consideration of time. These methods rely on the structural evolution of the 
graph, where the relationships and entities evolve over time, but the temporal aspects of the data are not explicitly modeled.

These methods are evaluated using performance metrics such as Mean Reciprocal Rank (MRR) and Hits@K, which provide benchmarks for comparing the effectiveness of different KGC 
approaches.

\subsection{Methods for Dynamic Entity Alignment}

Entity Alignment (EA) in dynamic KGs is a challenging task because entities may change over time. The paper categorizes Entity Alignment methods into two main groups.

\textbf{Temporal EA Methods} take into account the temporal changes in entities and relationships. These methods focus on aligning entities across different KGs while considering the 
evolution of the entities over time.

\textbf{Temporal and Evolving EA Methods} consider both the temporal evolution of entities and the changes that occur in the relationships among those entities over time. 
These methods are more comprehensive, as they account for both the temporal and dynamic aspects of entity alignment.

These techniques are important to ensure that the KGs remain coherent and complete as they evolve over time.
    \section{Literature Review}

The integration of dynamic elements into knowledge graphs (KGs) has attracted considerable attention in recent research, with a focus on addressing the inherent limitations of static KGs. 
One of the most relevant recent contributions is from \cite{polleres2023}, which provides a comprehensive overview of Dynamic Knowledge Graphs (DKGs). 
This study primarily focuses on the formal definition of different categories of DKGs, emphasizing the evolving nature of knowledge in open KGs. 
It highlights the use of neurosymbolic methods, which combine symbolic reasoning with neural network-based learning to represent and reason over these dynamic knowledge structures. 
The paper discusses how these methods can be leveraged for downstream tasks such as Knowledge Graph Completion (KGC) and Entity Alignment (EA). 
This work provides a crucial foundation for understanding the challenges posed by evolving knowledge in dynamic settings, though its scope is limited by a lack of exploration into large-scale 
real-world applications of these methods. Expanding on this, further research could delve deeper into the practical scalability of neurosymbolic methods in dynamic KGs, particularly for handling
continuous updates in complex domains.

In the context of refining KGs, \cite{paulheim2016} offers a survey on knowledge graph refinement methods, including KGC and entity matching. While this paper mainly addresses static KGs, 
the principles it discusses are foundational for understanding how the quality of KGs can be improved in dynamic environments. It provides valuable insight into the challenges faced by KGs 
in terms of consistency, completeness, and accuracy. However, when extended to DKGs, the dynamic evolution of entities and relationships introduces additional complexity. Refining KGs in a 
dynamic setting requires addressing the challenges of maintaining consistency across time steps and ensuring the accuracy of evolving data. Future research could focus on adapting the refinement
methods in \cite{luo2024} to account for the temporal aspects of DKGs, thereby advancing the quality control processes for dynamic knowledge representations.

The study by \cite{wang2016} surveys various methods for knowledge graph embedding, offering a comprehensive categorization of static and temporal embedding techniques. The paper highlights the 
importance of learning representations that capture the structural relationships between entities and their associated facts. While this work provides a solid foundation for embedding techniques
in static KGs, it only briefly touches on their extension to DKGs. The integration of temporal information into embeddings remains a significant challenge, especially when entities and 
relationships evolve over time. Expanding upon this, research could explore hybrid embedding techniques that are specifically designed to handle the temporal dynamics in DKGs, thus ensuring 
that the embeddings reflect not only static but also time-varying data.

A related area of research, as discussed in \cite{gesese2020}, focuses on multimodal knowledge graph embeddings that incorporate literals such as text, numbers, and images. 
This paper explores how literal information can be embedded within KGs to provide a richer representation of knowledge. The study is crucial for understanding how unstructured data can be 
integrated with structured knowledge in KGs. However, in the case of DKGs, handling time-sensitive literals remains underexplored. Future work could investigate how temporal aspects of literal 
data, such as changes in numerical values or the evolution of text-based descriptions, can be effectively captured in multimodal embeddings for dynamic KGs. This could open up new ways of 
handling real-world data, where the literal data is often subject to change over time.

In the same manner, \cite{alam2024semantical} explores semantically enriched embeddings for KG completion, focusing on the integration of background knowledge from large language models. 
This research takes a significant step toward improving the quality of KG completion by incorporating additional context and background information. While this approach is promising for 
static KGs, the dynamics of temporal changes in knowledge are not adequately addressed. The extension of this framework to dynamic KGs could offer new methods for improving KG completion tasks
in evolving contexts, especially by incorporating not only background knowledge, but also temporal relationships between entities. Exploring the interaction between semantic enrichment and 
temporal evolution could be a useful direction for future research.

Lastly, \cite{cai2023} provides an extensive survey of Temporal Knowledge Graph (TKG) completion methods, discussing the unique challenges of temporal reasoning within KGs. 
This paper contrasts Temporal KGs (TKGs) with DKGs, formally defining their differences and offering a comprehensive overview of representation learning techniques for these types of KGs. 
It highlights the complexities involved in representing time-sensitive information and proposes various approaches for temporal KG completion. Although it offers valuable insights into TKGs, 
there remains a gap in understanding how these methods can be adapted to DKGs, where knowledge evolves continuously rather than merely being time-stamped. Bridging the gap between TKG completion
methods and DKGs could lead to innovative solutions for handling dynamic, time-evolving knowledge representations.

By combining these insights, it becomes evident that while substantial progress has been made in the field of dynamic and temporal knowledge graphs, there are still many open questions and 
opportunities for further research. For instance, the integration of multimodal data, the scalability of neurosymbolic methods, and the adaptation of existing KG refinement techniques to dynamic
contexts represent key areas that can further advance the field. Additionally, research could focus on developing hybrid approaches that combine the strengths of various methods, such as 
embedding techniques with temporal data handling, to address the challenges posed by DKGs more effectively.
    \section{Research Proposals}

The research presented in the article provides a solid foundation for understanding and addressing the complexities of Dynamic Knowledge Graphs (DKGs), particularly in terms of representation 
learning and tasks like Knowledge Graph Completion (KGC) and Entity Alignment (EA). However, there remain significant challenges and gaps in the current methods that, if addressed, could 
significantly improve the adaptability and performance of DKG algorithms in real-world scenarios. If I were to further investigate this topic, my proposal would build upon the existing work 
while exploring new research questions, experiments, and practical applications that could extend and enhance the state-of-the-art in this field.

\subsection{Incorporating Schematic and Ontological Information}

One key limitation of the current approaches discussed in the article is their heavy reliance on the triple structure and graph-based representations, often overlooking the rich schematic 
or ontological information that can provide deeper insights into the relationships between entities and the structure of the knowledge itself. Ontologies and schemas are crucial for capturing 
the semantics of the relationships in a more structured manner, and they can provide a more robust framework for understanding dynamic changes in knowledge over time. 

My proposal would involve integrating ontological models into DKGs, enabling more sophisticated reasoning and inference mechanisms that consider the semantic meaning and relationships at a 
higher level of abstraction. This could lead to more accurate and interpretable predictions in KGC and EA tasks, especially in domains that involve complex, multi-layered knowledge structures 
such as healthcare, legal reasoning, and scientific research.

\subsection{Multimodal Data Integration}

Another significant gap identified in the literature is the lack of effective handling of literal information within KGs. While recent studies such as \cite{alam2024semantical} 
have made strides in incorporating multimodal data (e.g., text, images, and numbers) into static KGs, the temporal dimension of these literals has been largely unexplored. Multimodal data, 
especially when combined with temporal context, can provide richer, more nuanced information that can improve the quality of KG completion and entity alignment tasks. For instance, in the 
context of news articles or scientific research, entities and relationships evolve over time, and the associated textual, numeric, or image-based literals (e.g., facts, dates, experimental data)
also change. 

I propose developing a framework that incorporates both multimodal literals and their temporal evolution within DKGs. This could involve using neural architectures capable of processing and 
embedding multimodal information in a way that accounts for both the static and dynamic nature of KGs, improving their ability to represent real-world knowledge.

\subsection{Leveraging Large Language Models (LLMs) for Temporal Reasoning}

The use of Large Language Models (LLMs) in Knowledge Graph Completion (KGC) has attracted significant attention, as these models have shown the potential to enhance dynamic and temporal
reasoning in Knowledge Graphs (KGs). In particular, LLMs offer an avenue for improving Temporal Knowledge Graph Completion (TKGC), which involves predicting missing facts in knowledge graphs
that evolve over time. However, while promising, their application to temporal reasoning faces several challenges, as highlighted in a recent study \cite{luo2024}, which introduced a novel 
framework for applying LLMs to TKGC tasks.

The study explored how LLMs can be employed for both learning and forecasting temporal relationships within KGs, a task that is pivotal for tasks such as event prediction and forecasting future
states of entities. The authors propose a chain-of-history framework where LLMs predict future events by leveraging historical temporal data. Despite the potential of these models, the article 
points out that LLMs still struggle with the complexity of temporal reasoning, particularly in few-shot learning scenarios. The models often fall short of achieving robust temporal consistency, 
especially when predictions are made about future states that are not explicitly supported by prior data.

One critical issue identified in the article is the phenomenon of "hallucinations", where LLMs generate predictions that are temporally inconsistent or entirely fabricated. In the context of 
TKGC, hallucinations can be highly problematic, as the model may output facts that conflict with the actual historical or temporal data present in the graph. For example, an LLM might 
incorrectly predict that an entity has already completed an event in the future, which violates the temporal sequence of facts. Given the evolving nature of knowledge in dynamic KGs, ensuring 
that predictions remain grounded in reality is a key challenge.

The study also acknowledged that the few-shot learning capabilities of LLMs, while useful in certain contexts, have not led to significant improvements in performance for TKGC. The few-shot 
paradigm typically works well when there is a clear and abundant context to draw from, but in temporal reasoning tasks, especially those involving dynamic or incomplete data, LLMs often fail 
to capture the complex dependencies between past, present, and future states of knowledge.

In addressing these challenges, my proposal extends the work by investigating verification mechanisms that can help mitigate the risk of hallucinations. These mechanisms could involve 
cross-referencing predictions made by LLMs with existing data in the temporal context of a DKG. For instance, temporal coherence checks could be integrated, where a predicted temporal 
relationship is verified against the evolution of past states to ensure consistency. This would not only improve the accuracy of predictions but also ensure their reliability within the 
temporal constraints of the knowledge graph.

Furthermore, I propose exploring the integration of attention mechanisms within LLMs that focus on relevant temporal contexts. Attention mechanisms, which have shown success in many areas of 
NLP, could be adapted to prioritize facts and relationships that are temporally relevant, helping the model reduce hallucinations by ensuring that it is working within the bounds of accurate 
temporal data. This would be especially beneficial when dealing with large and complex datasets where temporal nuances are crucial to the success of TKGC tasks.

Another important consideration is that LLMs need to handle dynamic knowledge effectively, which involves incorporating a broader understanding of how knowledge evolves over time. While the 
article primarily focused on forecasting future events based on historical data, I propose extending this idea to create a more comprehensive framework for modeling dynamic change. This would 
involve enhancing LLMs to better understand and predict the dynamic evolution of entities and relationships within a DKG, thereby improving their ability to handle both temporal and 
non-temporal changes.

Lastly, the study emphasizes the potential of LLMs for forecasting and event prediction but also points out the limitations in current methodologies. The article, while offering a solid 
foundation for using LLMs in TKGC, reveals the need for more research into addressing the issues of model robustness and the integration of temporal reasoning into the learning process. 
My proposal would focus on enhancing the existing framework by incorporating advanced temporal reasoning techniques that are better suited to real-world dynamic environments.

\subsection{Real-World Application and Scalability}

One of the significant challenges in applying DKGs to real-world applications is the scalability of the algorithms, particularly when dealing with billions or trillions of triples. 
While the current approaches largely focus on theoretical or controlled experimental settings, they often overlook the computational complexities that arise when scaling to large datasets. 

My proposal would be to investigate the scalability of DKG algorithms, particularly in terms of both memory and processing power. This could involve exploring techniques such as distributed 
learning, graph partitioning, and parallel processing to handle the massive size of real-world knowledge graphs. Additionally, I would propose developing benchmarking frameworks that evaluate 
the performance of DKG algorithms across different domains, from social media data to scientific knowledge, to ensure that these algorithms can be generalized and effectively deployed in 
real-world scenarios.

\subsection{Cross-Domain Transferability of DKG Methods}

Another promising direction for future research is the transferability of DKG methods across different domains. Many current studies focus on specific applications or datasets, which limits 
the generalization of the methods. However, knowledge graphs are inherently flexible and can be applied to a wide range of domains, from finance to healthcare to social sciences. 

I propose exploring the transferability of DKG algorithms across various domains by developing universal, domain-agnostic models that can be adapted to specific tasks with minimal retraining 
or modification. This would involve investigating domain-specific knowledge and identifying key factors that could facilitate the transferability of algorithms while ensuring accuracy and 
scalability.

\subsection{Evaluation Approaches and Metrics}

The current literature tends to evaluate DKG methods based on standard benchmarks, but these benchmarks may not fully capture the intricacies of real-world knowledge graph tasks.

My proposal would include the development of new evaluation metrics that better assess the performance of DKG algorithms in dynamic environments. For example, in addition to traditional 
measures such as precision, recall, and F1-score, it would be valuable to incorporate metrics that evaluate the temporal consistency of predictions over time, the robustness to noisy data, 
and the ability to handle continuous changes in the knowledge base. Moreover, I would suggest conducting longitudinal studies to assess the long-term performance of DKG algorithms, 
particularly in settings where the knowledge graph is continuously updated and expanded.

    \section{Conclusion}

To conclude, this paper examines the challenges and complexities of Dynamic Knowledge Graphs (DKGs), focusing on Knowledge Graph Completion (KGC), and Entity Alignment (EA) tasks. 
While current methods offer valuable contributions, there are still considerable opportunities for further advancements. Incorporating ontological models, addressing the temporal aspects of 
multimodal data, and utilizing Large Language Models (LLMs) for temporal reasoning could greatly improve the performance and applicability of DKGs. Further research in these areas has the 
potential to significantly enhance the effectiveness and scalability of DKG algorithms in practical settings.

    \bibliographystyle{splncs04}
    \bibliography{references}
\end{document}