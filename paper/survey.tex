\section{Literature Review}

The integration of dynamic elements into knowledge graphs (KGs) has attracted considerable attention in recent research, with a focus on addressing the inherent limitations of static KGs. 
One of the most relevant recent contributions is from \cite{polleres2023}, which provides a comprehensive overview of Dynamic Knowledge Graphs (DKGs). 
This study primarily focuses on the formal definition of different categories of DKGs, emphasizing the evolving nature of knowledge in open KGs. 
It highlights the use of neurosymbolic methods, which combine symbolic reasoning with neural network-based learning to represent and reason over these dynamic knowledge structures. 
The paper discusses how these methods can be leveraged for downstream tasks such as Knowledge Graph Completion (KGC) and Entity Alignment (EA). 
This work provides a crucial foundation for understanding the challenges posed by evolving knowledge in dynamic settings, though its scope is limited by a lack of exploration into large-scale 
real-world applications of these methods. Expanding on this, further research could delve deeper into the practical scalability of neurosymbolic methods in dynamic KGs, particularly for handling
continuous updates in complex domains.

In the context of refining KGs, \cite{paulheim2016} offers a survey on knowledge graph refinement methods, including KGC and entity matching. While this paper mainly addresses static KGs, 
the principles it discusses are foundational for understanding how the quality of KGs can be improved in dynamic environments. It provides valuable insight into the challenges faced by KGs 
in terms of consistency, completeness, and accuracy. However, when extended to DKGs, the dynamic evolution of entities and relationships introduces additional complexity. Refining KGs in a 
dynamic setting requires addressing the challenges of maintaining consistency across time steps and ensuring the accuracy of evolving data. Future research could focus on adapting the refinement
methods in [6] to account for the temporal aspects of DKGs, thereby advancing the quality control processes for dynamic knowledge representations.

The study by \cite{wang2016} surveys various methods for knowledge graph embedding, offering a comprehensive categorization of static and temporal embedding techniques. The paper highlights the 
importance of learning representations that capture the structural relationships between entities and their associated facts. While this work provides a solid foundation for embedding techniques
in static KGs, it only briefly touches on their extension to DKGs. The integration of temporal information into embeddings remains a significant challenge, especially when entities and 
relationships evolve over time. Expanding upon this, research could explore hybrid embedding techniques that are specifically designed to handle the temporal dynamics in DKGs, thus ensuring 
that the embeddings reflect not only static but also time-varying data.

A related area of research, as discussed in \cite{gesese2020}, focuses on multimodal knowledge graph embeddings that incorporate literals such as text, numbers, and images. 
This paper explores how literal information can be embedded within KGs to provide a richer representation of knowledge. The study is crucial for understanding how unstructured data can be 
integrated with structured knowledge in KGs. However, in the case of DKGs, handling time-sensitive literals remains underexplored. Future work could investigate how temporal aspects of literal 
data, such as changes in numerical values or the evolution of text-based descriptions, can be effectively captured in multimodal embeddings for dynamic KGs. This could open up new ways of 
handling real-world data, where the literal data is often subject to change over time.

In the same manner, \cite{alam2024semantical} explores semantically enriched embeddings for KG completion, focusing on the integration of background knowledge from large language models. 
This research takes a significant step toward improving the quality of KG completion by incorporating additional context and background information. While this approach is promising for 
static KGs, the dynamics of temporal changes in knowledge are not adequately addressed. The extension of this framework to dynamic KGs could offer new methods for improving KG completion tasks
 in evolving contexts, especially by incorporating not only background knowledge, but also temporal relationships between entities. Exploring the interaction between semantic enrichment and 
 temporal evolution could be a useful direction for future research.

Lastly, \cite{cai2023} provides an extensive survey of Temporal Knowledge Graph (TKG) completion methods, discussing the unique challenges of temporal reasoning within KGs. 
This paper contrasts Temporal KGs (TKGs) with DKGs, formally defining their differences and offering a comprehensive overview of representation learning techniques for these types of KGs. 
It highlights the complexities involved in representing time-sensitive information and proposes various approaches for temporal KG completion. Although it offers valuable insights into TKGs, 
there remains a gap in understanding how these methods can be adapted to DKGs, where knowledge evolves continuously rather than merely being time-stamped. Bridging the gap between TKG completion
methods and DKGs could lead to innovative solutions for handling dynamic, time-evolving knowledge representations.

By combining these insights, it becomes evident that while substantial progress has been made in the field of dynamic and temporal knowledge graphs, there are still many open questions and 
opportunities for further research. For instance, the integration of multimodal data, the scalability of neurosymbolic methods, and the adaptation of existing KG refinement techniques to dynamic
contexts represent key areas that can further advance the field. Additionally, research could focus on developing hybrid approaches that combine the strengths of various methods, such as 
embedding techniques with temporal data handling, to address the challenges posed by DKGs more effectively.