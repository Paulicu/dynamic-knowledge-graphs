\section{Literature Review}

The integration of dynamic elements into knowledge graphs (KGs) has attracted considerable attention in recent research, with a focus on addressing the inherent limitations of static KGs. One of the most relevant recent contributions is from \cite{polleres2023}, which provides a comprehensive overview of Dynamic Knowledge Graphs (DKGs). This study primarily focuses on the formal definition of different categories of DKGs, emphasizing the evolving nature of knowledge in open KGs. It highlights the use of neurosymbolic methods, which combine symbolic reasoning with neural network-based learning to represent and reason over these dynamic knowledge structures. The paper discusses how these methods can be leveraged for downstream tasks such as Knowledge Graph Completion (KGC) and Entity Alignment (EA). This work provides a crucial foundation for understanding the challenges posed by evolving knowledge in dynamic settings, though its scope is limited by a lack of exploration into large-scale real-world applications of these methods. Expanding on this, further research could delve deeper into the practical scalability of neurosymbolic methods in dynamic KGs, particularly for handling continuous updates in complex domains.

In the context of refining KGs, \cite{paulheim2016} offers a survey on knowledge graph refinement methods, including KGC and entity matching. While this paper mainly addresses static KGs, the principles it discusses are foundational for understanding how the quality of KGs can be improved in dynamic environments. It provides valuable insight into the challenges faced by KGs in terms of consistency, completeness, and accuracy. However, when extended to DKGs, the dynamic evolution of entities and relationships introduces additional complexity. Refining KGs in a dynamic setting requires addressing the challenges of maintaining consistency across time steps and ensuring the accuracy of evolving data. Future research could focus on adapting the refinement methods in [6] to account for the temporal aspects of DKGs, thereby advancing the quality control processes for dynamic knowledge representations.

The study by \cite{wang2016} surveys various methods for knowledge graph embedding, offering a comprehensive categorization of static and temporal embedding techniques. The paper highlights the importance of learning representations that capture the structural relationships between entities and their associated facts. While this work provides a solid foundation for embedding techniques in static KGs, it only briefly touches on their extension to DKGs. The integration of temporal information into embeddings remains a significant challenge, especially when entities and relationships evolve over time. Expanding upon this, research could explore hybrid embedding techniques that are specifically designed to handle the temporal dynamics in DKGs, thus ensuring that the embeddings reflect not only static but also time-varying data.
